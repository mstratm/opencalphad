\documentclass[12pt]{article}
\textwidth 165mm
\textheight 210mm
\oddsidemargin  1mm
\evensidemargin  1mm
\topmargin 1mm
\usepackage[latin1]{inputenc}
\pagestyle{empty}

\begin{document}

{\bf \Large Why and how to use getkey}

\bigskip

Updated \today

\bigskip

When running programs on LINUX and similar OS there is no editing or
history available on the command level.  On Windows there is normally
a possibility to recall previous commands and to edit the command
line.

In order to have this facility it is necessary for a program to read
character by character from the keyboard without waiting for the user
to press the RETURN key.  Such a facility is provided by the GETKEY
function written in C.

Such a routine can also be used for other purposes for example to
allow users to interrupt lenghty calculations or listings without
leaving the program and loosing results.

The getkey function used here has been written by Urban Jost and has
been downloaded from the web
http://www.urbanjost.altervista.org/LIBRARY/libCLI/Getkey/getkey.html

Each OS and terminal can have different ways to navigate the cursor on
the screen.  At the end of this text there is an extract of the bintxt
subroutine that is found in the metlib3.F90 file.  You can modify the
way to control your editing and you may have to modify the code that
moves the cursor one step backward on on your screen.  The way to edit
input I have implemented follows mainly the emacs standard.

\bigskip

{\bf To install getkey these are the necessary steps:}

\begin{enumerate}

\item The C routine getkey() must be compiled with C for your OS.
Available are BSD, V13, V13B, LINUX.  On a mac use BSD, otherwise you
have to test which does not give any error messages.

\begin{verbatim}
>>> On MAC that is: gcc -c -DBSD getkey-original+testprogram.c
\end{verbatim}

If you wish you can use the getkey testprogram, you must then compile
with the following options:

\begin{verbatim}
>>> On MAC that is: gcc -DBSD -DTESTPRG getkey-original+testprogram.c
\end{verbatim}

This creates a progran a.out.  It is useful to check which code that
moves the cursor backward (BACKSPACE) on your screen.

\begin{verbatim}
>>> On MAC that is: ./a.out
press keys ('q' to quit)
C:KEY=j 106
C:KEY=k 107
C:KEY=n 110
C:KEY=27
C:KEY= 2
C:KEY= 127
C:KEY= 3
C:KEY=c 99
C:KEY=q 113
\end{verbatim}

\item When you managed to compile (and test) rename the compiled
version of getkey (without the test program) to just getkey.o

\begin{verbatim}
>>> On MAC that is: gcc -c -DBSD getkey-original+testprogram.c
>>> Om MAC that is: mv getkey-original+testprogram.o getkey.o
\end{verbatim}

Getkey use the Fortran/C iso-C-binding through the interface defined
in M\_getkey+testprogram.F90.  You must use a Fortran compiler
compliant with the 2003 Fortran standard like GNU Fortran 4.8 (or
later).  Compiling M\_getkey+testprogram.F90 creates a module
m\_getkey.mod which is necessary for linking the getkey function to
the OC software.

\begin{verbatim}
>>> On MAC that is: gfortran -c M_getkey+testprogram.F90
\end{verbatim}

You can again compile with a small test program:

\begin{verbatim}
>>> On MAC that is: gfortran -DTESTPRG0 M_getkey+testprogram.F90 getkey.o
\end{verbatim}

You have now a new a.out you can test.

\begin{verbatim}
>>> On MAC that is: ./a.out
 begin striking keys to demonstrate interactive raw I/O mode
 q to quit; up to 40 tries allowed
           1  f03:key=b->          98
           1  f03:key=->           2
           1  f03:key=B->          66
           1  f03:key=          27
           1  f03:key=q->         113
\end{verbatim}

\item In OC a subroutine routine bintxt is used for all input.  This
has a plain "read" statement to obtain input from keyboard or macro
files.  In the special metlib3+getkey.F90 provided in this directory
the getkey function is used for keyboard input.

You may change the characters you want to use for line editing and you
MUST check (using the test programs above) which code moves the cursor
backward one step on the screen.  Then look in the metlib3+getkey.F90
file around lines 3030 to 3260.  You will find a section like:

\begin{verbatim}
! CONTROL CHARACTERS FROM KEYBOARD
! DEL delete current character
    integer, parameter :: ctrla=1        ! CTRLA move cursor to first position
    integer, parameter :: backspace2=2   ! CTRLB move cursor one step left
    integer, parameter :: ctrlc=3        ! CTRLC terminate program
    integer, parameter :: ctrld=4        ! CTRLD delete char at cursor
    integer, parameter :: ctrle=5        ! CTRLE move cursor to last position
    integer, parameter :: forward=6      ! CTRLF move cursor one step right
    integer, parameter :: HELP=8         ! CTRLH give coordinates and update
    integer, parameter :: TAB=9          ! CTRLI end of input
    integer, parameter :: ctrlk=11       ! CTRLK delete to end of line
    integer, parameter :: return=13      ! CTRLM end of input
    integer, parameter :: DEL=127        ! DEL delete char left of cursor
    integer, parameter :: mode=17        ! CTRLQ toggle insert/replace
! on MAC same as arrow UP DOWN FORWARD suck
    integer, parameter :: backspace=27   ! CTRL[ 
!--------------------  
! UP previous history line (if any)
! DOWN and LF next history line (if any)
    integer, parameter :: CTRLP=16       ! CTRLP previous in history
!    integer, parameter :: UP=27         ! uparrow previous in history
    integer, parameter :: LF=10          ! CTRLJ next in history
!--------------------
! backspace on a MAC screen
    integer, parameter :: tbackspace=8
!-----------
\end{verbatim}

The most important is the value for "tbackspace".

The other control characters you can change to whatever you prefer.
The selection here is close to emacs style editing.

On a Mac all the arrow keys give the value "27" by getkey so
unfortunately they will all be interpreted as backspace.  If they give
different codes on you computer you can add these as well as any other
preferences you have for online editing to the bintxt subroutine.

You should keep a copy of the original metlib3.F90 until you are
satisfied with the editing and do not have the cursor jumping all over
the screen.

There is a small history kept by bintxt, max 20 lines that can be
recalled using "ctrl-P".  If you have used several "ctrl-P" you can
then use "ctrl-J" to recall the history lines you already passed.  It
is the possibility to repeat the same command several times that I use
most with this facility.

\item When you have edited the metlib3+getkey.F90 file you can
rename it to just metlib3.F90 and then compile and link the whole OC
software with this facility using the makefile Makefile+getkey that
linkes the sequential version.  To run a makefile that is not called
Makefile give make -f

\begin{verbatim}
>>> On Mac that is: make -f Makefile+getkey clean
>>> On Mac that is: make -f Makefile+getkey
\end{verbatim}


You have to use "clean" each time you want to recompile because OC has
its source code on several directories and one has to do special
things to force "make" to understand that it has to recompile all.

In the Makefile+getkey the module file with the interface m\_getkey.mod
and the compiled getkey.o are copied from the utilities/GETKEY/
directory.

You can modify yourself the Makefile-parallel to include getkey.  I do
not like Makefiles so I have many of them.
\end{enumerate}

Mave fun and make OC better!

\end{document}


\begin{verbatim}
\end{verbatim}




